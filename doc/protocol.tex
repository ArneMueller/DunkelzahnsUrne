\documentclass[a4paper,10pt]{article}

\usepackage[utf8]{inputenc}
\usepackage[ngerman]{babel}
\usepackage{amsmath}
\usepackage{amssymb}
\usepackage{fancyhdr}
%\usepackage{graphicx}
\usepackage{geometry}
\usepackage{enumerate}
%\usepackage{listings}
%\usepackage[rflt]{floatflt}
%\usepackage{tikz}
%\usepackage{wrapfig}
%\usepackage[standard, thmmarks]{ntheorem}
%\usepackage{algorithm}
%\usepackage{algorithmic}
%\usepackage{algorithm2e}
%\usepackage{subfigure}
%\usepackage{mathbbol}
\usepackage{listings}
\usepackage[pdfborder={0 0 0}]{hyperref}


%\lstset{numbers=left, numberstyle=\tiny, basicstyle=\footnotesize, numbersep=5pt}
%\lstset{extendedchars,inputencoding=utf8}
%\lstset{tabsize=2,breaklines}
%\lstset{language=}
%\renewcommand{\lstlistingname}{Quelltext}

\geometry{left=2cm, top=3cm, right=2cm, bottom=2.5cm}

%\renewcommand{\headrulewidth}{0.4pt} 
%\renewcommand{\footrulewidth}{0.4pt}

\pagestyle{fancy}

%\newcommand{\qed}{\hfill \ensuremath{\Box}}

\newcommand{\wiki}[2]{\href{http://wiki.piratenpartei.de/#1}{#2}}
\newcommand{\mytitle}{ Die Akkreditierungsurne }


%opening
\title{\mytitle}
\author{AG Liquid Democracy, Ansprechpartner: \wiki{Benutzer:Wobble}{Wobble}, \wiki{Benutzer:Dunkelzahn}{Dunkelzahn}, \wiki{Benutzer:moonopool}{moonopool}}

\lhead{Vorschlag der AG Liquid Democracy}
\chead{\mytitle}
\rhead{\today}

\newcommand{\Base}{\texttt{Base64}}
\newcommand{\Byte}{\texttt{Binär}}
\newcommand{\RSA}{\href{https://tools.ietf.org/html/rfc3447}{RSA }}
\newcommand{\SHA}{\href{http://csrc.nist.gov/publications/fips/fips180-2/fips180-2withchangenotice.pdf}{SHA-512 }}


\begin{document}

\parindent 0em
\parskip 0.5em

% Umfang: 5 Seiten!

\maketitle

\tableofcontents

\section{Einleitung}
Dieses Dokument definiert formal welche Komponente wie auf Anfragen etc. reagieren muss bzw. kann. In Abschnitt \ref{sec:datentypen} werden die Datentypen, die zum Informationsaustausch genutzt werden definiert. In Abschnitt \ref{sec:verfahren} werden die einzelnen Verfahrensschritte definiert. 

Dabei werden folgende Schlüsselworte (in Großbuchstaben) genutzt:
\begin{itemize}
 \item MUSS beschreibt eine Anforderung, die unbedingt zu erfüllen ist. Verletzungen von ``MUSS-Anforderungen`` müssen nicht verhinderbar aber durch dritte dokumentierbar sein.
 Eine Nichterfüllung kann zur Bestrafung des Betreibers der Komponente führen.
 \item SOLL beschreibt eine Anforderung, die für einen ordnungsgemäßen Ablauf erforderlich ist. Eine Verletzung einer ``SOLL-Anforderung'' muss nicht durch dritte dokumentierbar sein, wird aber in der Regel zu einer Fehlermeldung in einem Nachfolgeschritt des Ablaufes führen. Verletzungen von ``SOLL-Anforderungen'' sind nicht zu bestrafen.
 \item SOLL NICHT beschreibt eine Aktion, durch deren Umsetzung Eigenschaften des Verfahrens für den Komponentenbetreiber verloren gehen aber keine Auswirkungen auf andere Komponenten haben. Verletzungen von ``SOLL NICHT-Anforderungen'' sind nicht zu bestrafen.
 \item DARF NICHT beschreibt eine Aktion, die wesentliche Eigenschaften des Systems gefährdet. Verletzungen von ``DARF NICHT''-Anforderungen können bestraft werden.
\end{itemize}


\section{Datentypen} \label{sec:datentypen}
Zur Kommunikation werden nur Text-Nachrichten verwendet. Alle Text-Nachrichten sind in \href{https://tools.ietf.org/html/rfc3629}{UTF-8} codiert. Zeilenumbrüche sind Linux-codiert, d.h. nur mit LF. Implementierungen dürfen aber auch Windows-Zeilenumbrüche tolerieren, solange sie diese im Nachrichtentext zu Linux-Zeilenumbrüche umcodieren.

\subsection{Abkürzungen}
\begin{itemize}
 \item \Base-Codierung von Binärdaten. Binärdaten werden durch Strings gemäß \href{http://www.ietf.org/rfc/rfc2045.txt}{RFC 2045}, Section 6.8. Base64 Content-Transfer-Encoding codiert. Die Codestrings enthalten keine Zeilenumbrüche.
 \item \Byte-Codierung von Zahlen. Die Binärdarstellung einer Zahl ist die 2-Komplement Darstellung in Big-Endian Bytereihenfolge. 
 \item \Base-Codierung von Zahlen. Die \Base-Codierung einer Zahl ist die \Base-Codierung der \Byte-Codierung der Zahl. 
\end{itemize}

\subsection{Liste verifizierter Accounts}
Jede Zeile der Liste verifizierter Accounts beginnt mit ``Account: '' und dann folgt ein \Base-codierten Hash.

Jeder Hash, der so in einer Liste verifizierter Accounts $L$  eingetragen ist, gilt als \emph{verifiziert nach $L$}.

\paragraph{Beispiel:}
\begin{verbatim}
Account: ic/ZTBgESfPpRL6EnoOVEJNN5a4PvFNotQiIJAbZf7PxG8eseIn5nwshybMmT6W9ykIWzxQaOml7L2Q4VkwXPA==
Account: uEiJDKYXVlS71A+OfbHK6S7r0SmxnK/lN8sJhE2WJsUEX5+F+WPQckFUBT2mmalszwRxutKUiHIqCDc5a1GqjQ==
\end{verbatim}


\subsection{Öffentlicher Schlüssel} \label{sec:publicKey}
Ein Öffentlicher Schlüssel im Sinne dieser Spezifikation ist ein öffentlicher RSA Schlüssel gemäß \href{https://tools.ietf.org/html/rfc3447}{PKCS \#1}. 
Demnach besteht ein öffentlicher Schlüssel aus zwei Zahlen: einem Modulus und einem öffentlichen Exponenten.

Die Codierung besteht aus zwei Zeilen.
Die erste Zeile beginnt mit ``Modulus: '' und dann folgt der \Base-codierte Modulus des Schlüssels.
Die zweite Zeile beginnt mit ``Exponent: '' und dann folgt der \Base-codierte Exponent des öffentlichen Exponenten.

\paragraph{Besipiel:}
\begin{verbatim}
Modulus: AIe9uKt9sgj1mwNtuZDUPad1q+iaMKpTC1lTWN9ytA/NfJIEe8/A+wYuMgVWkQDX4P1zFBekgixqqcr9TB6xK0Kdg7ivvZfu5UfYqkRXJ3ehe5gS0Ip0sfMTRSO2Rualy3OQXUr2NA7AbeO5aMtwsrZ5R0CSvgBGW4YNxJYiqjVGgJ1aAxIJvCyVLXMLwEAvkQoJhRf26K7FPdS22WnBzvcggv/QQgvDD817R/04AfTD3IBmouWXdse2SNuzpJqQPAmXd7jTrR2lMLMhWDVl2kbROO3i1oNhkels0IBjCk5WXLs8T/qK462fOrqdWU+1VHXRR+SX2VP2VO4rdbf4lNa4PyjOmx2LHv4oypv5lzUrMTkbJ+VSKjnzgopKO+qdmxTQ8qIImzXjdG+5TQOdbnuzcOw+CChu3qKfk/TPDLOeW5Asl1IOHvsnejpmZ6mtSxNETsgqGl/EWz+UNtnaIOI+9R7oTvP4tFih0RB4CKwyZHA3hjsVlN9p5FVuJTYoRouWe/seClnDiix92/OUe9nOvAHO8ogLgh2B8yThK0vSfsH77odDjnFG4DgKRF5QJMVXMebgtanH9Sn5m1Nl7NmOykk5Q0Ha8djjkNAH/6ii9QfvJYmEuWYU524VGEzKmP8qnwG5DNzk6w+nSxTbS4/6/Kb2FL8OQdQfc9HARZgT
Exponent: AQAB
\end{verbatim}


\subsubsection{Hash eines öffentlichen Schlüssels}
Sei $m$ die \Byte-Codierung des Modulus und sei $e$ die \Byte-Codierung des öffentlichen Exponenten des öffentlichen Schlüssels.
Der Hash des öffentlichen Schlüssels ist die \Base-Codierung des \SHA Hashes der Bytefolge $me$.

\subsection{Liste registrierter Accounts}
Jeder Eintrag der Liste registrierter Accounts besteht aus $3$ zusammenhängenden Zeilen. In der Liste registrierter Accounts sind nur solche Einträge enthalten.

Die erste Zeile jedes Eintrages beginnt mit ``Hash: ''gefolgt von einem \Base-codierten Hash.
Die zweite und dritte Zeile jedes Eintrages enthalten die Codierung eines öffentlichen Schlüssels.

Jeder Hash eines gültigen Eintrages in einer Liste registrierter Accounts $L$ gilt als \emph{registriert nach $L$}. 
Ein Eintrag mit Hash $h$ und öffentlichen Schlüssel $p$ ist gültig, falls der Hash von $p$ gleich $h$ ist.

\paragraph{Beispiel}
\begin{verbatim}
Hash: ic/ZTBgESfPpRL6EnoOVEJNN5a4PvFNotQiIJAbZf7PxG8eseIn5nwshybMmT6W9ykIWzxQaOml7L2Q4VkwXPA==
Modulus: AIR3IwoK65srUqdRJjDqJHaFPodRoNdYUMc+H3RfckyYLxs/YsLCcBXi8fh1h6kMLJR2+mWSo0GRAmfgh8iTCEUKDxq4VOH2ofU4P0MzY1+P1HtijdzssgfDgkkPp/fXlTReQyAC1Lzmscq2/lsDwADR7Erhxy2Ogy3NawvJOS4+uLwQXzTBcl94aJRk20zs6kyY15InVwCLDONIg/ywHXjC3F3gwhn2yzX/KoA2Lz785gznazv+f1vC4JnGSAyprykcTIuoVCkd7LR6Na45mJ8QRnKn0b+ZvphntWNOYKzk+AeZZX0Q5HgkzhtGbQcHO6TMoB8DqXW049aWOVM+jbx7hh8gF2lqLw42v94ESiU5pYRywlbX7oFtpMCPIneTnwG7HFhITA20HPnFi+5LClbNUq/abTFoaXG8eaFlv8bfaNLI0JPQkz25vpJ6geYHLy0O/tCxqGt3/9Sd3m8cz5VDf8RfwDJ9QQIG2Do0Ty40MnlCYHMP4kUfF1Gtq4S9vqgcPy8xYHaz9NqnjElkEPEhXiLip5Krbw9isebYUaKItayQ46hm6kFv7wBNZuST0etEwis0O2zy+eXsmzGo8DrD6PWe7FK1dZcV4d1kpCfMiuBnU8b4SBh8Q/bP4efI9XpMkea/T5wxTi8Rx0p8pclNSA2qudJVp3F5E4Prya75
Exponent: AQAB
Hash: i9YMFc+vWTpO3B5hxwsg80UvHX6sGUKdhyly7RQOfg230UZfaVfN6fqHMtfb3bxydpRUEuQvgKgW450FwzrSww==
Modulus: AM0Q3ZDAluR2N6K0j4FRu7mhAkgmX6yjxJQmORpYk9b8EcVtE2PHdYjy4Yd4LW+MLfmfNX55iU6+XRV91kHN9Z1G6P+G9Ob72EkMJ9PqKwT/vnQNJW8WSrfp1dpKpw7jACdvV0UE6lsvOXm+efvlc5jHh85BprAA/TU9hV/yTb/Aq819eatJHW6F2Wa9XibNzb0rNPExdpzRVlskKEF7U0IH52pnBDcu6UseYzYZW43GWC47nGdftelG/QhRv6YLk9DKQ6eujmOIsXsrN18xARxeomxIqjarUq+gfAJYSIwa5WUirk9hvrXD8OanQXn5iRdF+283hzUmFD5NyR5xL/DjDggb+5c9nuLkiLKSLuKkifdldj0G/G4QNg/56h3bmetYJRPKTKlgmy8EHfTvVWCRJ/LmcHeXIl1ZeOAhj9Phduh3tmq4J5r6qIM5uH6aq8JEaLqYxyXyGd5+sPVb5Ro1dE6R60OcCiso8WSmun4P9+QUxsFG+6iCYIWeQh2J15H1d+YcBn4wizO/mnFINBL7A6VfTssBj8ObmsJ/Mt5xHutseRt6eSe4o6HCvMtRRNdf/tmZ+aLehlFFwemahScW6XFwtOlifCP4XI2vSYbzE3SscdJuD/41t8dZjDdoimlRCU25ojAiP95afT9DsENQoU2hlOn12q3DvH680xfP
Exponent: AQAB
Hash: uEiJDKYXVlS71A+OfbHK6S7r0SmxnK/lN8sJhE2WJsUEX5+F+WPQckFUBT2mmalszwRxutKUiHIqCDc5a1GqjQ==
Modulus: AIe9uKt9sgj1mwNtuZDUPad1q+iaMKpTC1lTWN9ytA/NfJIEe8/A+wYuMgVWkQDX4P1zFBekgixqqcr9TB6xK0Kdg7ivvZfu5UfYqkRXJ3ehe5gS0Ip0sfMTRSO2Rualy3OQXUr2NA7AbeO5aMtwsrZ5R0CSvgBGW4YNxJYiqjVGgJ1aAxIJvCyVLXMLwEAvkQoJhRf26K7FPdS22WnBzvcggv/QQgvDD817R/04AfTD3IBmouWXdse2SNuzpJqQPAmXd7jTrR2lMLMhWDVl2kbROO3i1oNhkels0IBjCk5WXLs8T/qK462fOrqdWU+1VHXRR+SX2VP2VO4rdbf4lNa4PyjOmx2LHv4oypv5lzUrMTkbJ+VSKjnzgopKO+qdmxTQ8qIImzXjdG+5TQOdbnuzcOw+CChu3qKfk/TPDLOeW5Asl1IOHvsnejpmZ6mtSxNETsgqGl/EWz+UNtnaIOI+9R7oTvP4tFih0RB4CKwyZHA3hjsVlN9p5FVuJTYoRouWe/seClnDiix92/OUe9nOvAHO8ogLgh2B8yThK0vSfsH77odDjnFG4DgKRF5QJMVXMebgtanH9Sn5m1Nl7NmOykk5Q0Ha8djjkNAH/6ii9QfvJYmEuWYU524VGEzKmP8qnwG5DNzk6w+nSxTbS4/6/Kb2FL8OQdQfc9HARZgT
Exponent: AQAB
\end{verbatim}


\subsection{Nachricht}
Die erste Zeile jeder Nachricht ist ``Author: '' gefolgt von einem \Base-codierten Hash $h$. \\
Die zweite Zeile jeder Nachricht ist ``Digest: '' gefolgt von einem \Base-codierten Digest $d$. \\
Die dritte Zeile jeder Nachricht ist ``Signature: '' gefolgt von einer \Base-codierten Signatur $s$. \\
Danach kann beliebiger Text $t$ folgen, der mit LF enden muss. $t$ wird auch der Text der Nachricht genannt.

Eine Nachricht ist \emph{gültig} gegenüber einer Liste registrierter Accounts $R$, falls
\begin{itemize}
 \item der Hash $h$ nach $R$ registriert ist.
 \item der Digest $d$ gleich der \Base-codierung des \SHA Hashes von $t$ ist und
 \item die Signatur $s$, entschlüsselt mit dem Schlüssel, der in $R$ für den Hash $h$ gespeichert ist, gleich $d$ ist. Als Verschlüsselungs-Algorithmus muss \RSA mit PKCS v1.5 Padding verwendet werden.
\end{itemize}

Eine Nachricht ist \emph{verifiziert} gegenüber einer Liste verifizierter Accounts $V$, falls der Hash $h$ nach $V$ verifiziert ist.

\paragraph{Beispiel:} siehe Account-Erstellungs-Nachricht.

\subsection{Account-Erstellungs-Nachricht}
Eine Nachricht ist eine gültige Account-Erstellungs-Nachricht, falls 
\begin{itemize}
\item die erste Zeile des Nachrichtentextes ``Publish Account'' ist,
\item die zweite Zeile des Nachrichtentextes muss mit ``Hash: ''beginnen, gefolgt von einem \Base-codierten Hash $h$,
\item $h$ mit dem Hash, der bei Author angegeben ist, übereinstimmt,
\item die dritte und vierte Zeile des Nachrichtentextes die Codierung eines öffentlichen Schlüssels $p$ sind,
\item $h$ der Hash von $p$ ist,
\item der Digest $d$ gleich der \Base-codierung des \SHA Hashes des Nachrichtentextes ist und
\item die Signatur $s$, entschlüsselt mit $p$, gleich $d$ ist. Als Verschlüsselungs-Algorithmus muss \RSA mit PKCS v1.5 Padding verwendet werden.
\end{itemize}

\paragraph{Beispiel:}
\begin{verbatim}
Author: hpO/rM/QjX07fGibrUbGtqr7+OcRFda5jaCn2xLpqdt4WmUYphmS/P4ON3GNofjh8fWpOKe+nxDD/zgDC3X6rQ==
Digest: PXiH1qZOKEtCoV07j88e38USVok5ehtc7j6OvOy7oJGCGLibCzrvEE91V32cSQYvvSGBXU/f2Obxs4Zkcubk9g==
Signature: CipK+znGV1Dg2S8srorhDTKLPX18EGYKaEAwvVBknUfgTlmgn69rd3vxy7xV+jJkp+8VRX9XW2PtAJBVNbF+LastUE9vadQeVXNyMhhFx0foRADa7FPGF6ZmUWA/FPGi3kX4LhxDum2wrxoo7CgV2MYVOH9nM2Lv+5l/68LLHY0UgkseEarzR+xjPrIfYkCfiKIXZNCg6ZqjT5v5RLjytvfl1IfiHa7ua0LXaa6OoD7jOZivM7s/UfSaZ9KhbjWu4eag7xwM5TX2/GxuuXM+v30notXFLG1BsSywgse1E2FG7TezZB3cc5eHtXYS3hkeiqmPrnoZitqD3XAwEVi1LxQFHZS3Lx2YLclnPB8FwIdvFnJOVmoJbJPN89U4J6Z1WksX9yxhi5isqkeU0/qejYp7BMdqMjY9ABoAB96T3I7EG91sPAT6QfBUjT0Wd/TSQNGx6HEjJKkCvZJAmzgr9Pjl4DNZj2flpmNJOHs07lePVOklIEnkOZ4RrpPEC0VzNbj5ME7ZCeDj0rzgR4OLYgzb5bDj8qjqOX5m9RMC23b9OVLPuzLoFz0fud/57C0wdePTgQkB8HuGm2yJ2XlQj1Tbmvp3S5p12a8qGkjB3mdcIR9zvQw5PD/0Xe+jm3gUpFZmIsnysnXBVXFcOve3+ssdvCsPzrAl0vOZrofv66U=
Publish Account
Hash: hpO/rM/QjX07fGibrUbGtqr7+OcRFda5jaCn2xLpqdt4WmUYphmS/P4ON3GNofjh8fWpOKe+nxDD/zgDC3X6rQ==
Modulus: AINYDQjkC6KAQAst9hkBVpVaXIOYbBSic8QMzNqjzNjIgmTxoPfOtLLwiOL3FhB5lfCTVxulZ15PAxc1AClrPJMCSX9V7e8HKWnmE+SEvaQjMiudUbHuasRcaFknsJRIlyZXhLR6DW1uR7NK574gIuyawChXHJIrqAfSnfw5JxIc2VRwbwUNAwS4sOd/pm7lsEJwP9LVnVvlm1UDRvA1BBs++4KGimUn7w85xPkVqQHPBsjAoPaQEEvoN8W6CnVg0FhuTGRQurQQeKXYRfcQINNEj/W0AnqkzLAwNBwW5HaLHFoqJgiE97HRtl/M6y/6RfDcL7iU8ikkZPKhVwf7DWaAlmiZpYhsNGPdvPfuDQ6ZglOnFSQA+53NJE5y18Au4xx+FI7yTrZak8tBKC4ysNwxcL74YFSntU/UbJvw5qvgCV5bFLCBVNeBLye9qWmBJTaAjJ01H2ymTcM4T6ed5MQdeZINwpDFdcmZo1zjuZU+4wm/SHWDC1WgITmO2CFWY+vIZKOCNduDRpRPCoXdaPZLMoHe1JdilPqiFmfOG6kJBYq37eQ5BhraEHxz6NH7nUe/InSmxl3zIiTTe+Yht7V5f4mc2RvqGPSP1gdtDyEupcrNDR43j/FES0CheEPdl6vexFNTJT3MfjKbmML8d6Ql1ptuqiQkBS0fctt/SFad
Exponent: AQAB

\end{verbatim}


\subsection{Akkreditierungsbrief}
Der Akkreditierungsbrief ist ein auf Papier gedruckter Hash des öffentlichen Schlüssels. Konkret ist enthalten:
\begin{itemize}
 \item Der \Base-codierte Hash
 \item Der Hash als QR-CODE (TODO:genauer)
\end{itemize}


\section{Verfahren} \label{sec:verfahren}
Das System ist in die folgenden Komponenten aufgeteilt:
\begin{itemize}
 \item Verification-Server: Dieser stellt eine Liste der verifizierten Hashes bereit
 \item Account-Server: Dieser verwaltet die Liste aller registrierten Accounts
 \item Urne: Die Urne dient der Sammlung der Hashes der öffentlichen Schlüssel
 \item Client: Jeder Pirat betreibt einen Client
\end{itemize}

\subsection{Komponente: Verification-Server}
Der Verification-Server verwaltet die offizielle Liste der verifizierten Accounts, welche durch den Urnengang verifiziert wurde.
\subsubsection{statisches Verhalten}
\begin{itemize}
 \item Der Verification-Server MUSS auf einem wohldefinierten Ort die offizielle Liste der verifizierten Accounts veröffentlichen.
\end{itemize}


\subsection{Komponente: Account-Server}
Der Account-Server verwaltet die offizielle Liste der registrierten Accounts, welche sich bis jetzt registriert haben.
\subsubsection{statisches Verhalten}
\begin{itemize}
 \item Der Account-Server MUSS auf einem wohldefinierten Ort die offizielle Liste der registrierten Accounts veröffentlichen.
\end{itemize}

\subsubsection{Empfang einer Nachricht $m$ }
\begin{itemize}
 \item Der Account-Server MUSS prüfen, ob $m$ eine gültige Account-Erstellungs-Nachricht ist.
 \item Ist $m$ eine gültige Account-Erstellungs-Nachricht, so MUSS der Account-Server in der offiziellen Liste der registrierten Accounts einen Eintrag unter dem Hash von $m$ mit dem Schlüssel von $m$ erstellen. Falls schon ein Eintrag mit dem Hash von $m$ existiert, so MUSS der existierende Eintrag überschrieben werden.
 \item Ist $m$ keine gültige Account-Erstellungs-Nachricht, so DARF die offizielle Liste der registrierten Accounts NICHT modifiziert werden.
\end{itemize}

\paragraph{Begründung}
Nachrichten müssen auf Gültigkeit geprüft werden, damit dritte nicht die Account-Details Fremder überschreiben.

Durch den öffentlich sichtbaren Effekt der Nachricht (Aktualisierung der registrierten Accounts), ist keine weitere besondere Antwort-Nachricht nötig. Mmh... dann sollte diese offizielle Liste vom Account-Server signiert werden.


\subsection{Komponente: Client}
Mit dem Client können neue Accounts erstellt werden und Nachrichten versendet werden.
\subsubsection{Account erstellen}
\begin{itemize}
 \item Der Client erstellt einen neues Schlüsselpaar mit privaten Schlüssel $p$ und öffentlichen Schlüssel $q$.
 \item Der Client SOLL das Schlüsselpaar auf einem privaten aber zugänglichen Ort speichern.
 \item Der Client SOLL den privaten Schlüssel NICHT veröffentlichen.
 \item Der Client SOLL den öffentlichen Schlüssel NICHT in einer Form veröffentlichen, die es ermöglicht Schlussfolgerungen auf den Inhaber des öffentlichen Schlüssels zu ziehen.
\end{itemize}

\subsubsection{Account registrieren}
\begin{itemize}
 \item Der Client SOLL eine Account-Erstellungs-Nachricht aus dem öffentlichen Schlüssel $q$ erstellen und dem Account-Server senden. Die Nutzung von \href{https://www.torproject.org/}{Tor} wird empfohlen.
 \item Der Client SOLL überprüfen, ob der Account-Server die öffentliche Datenbank aktualisiert hat und eine Kopie speichern. Die Nutzung von \href{https://www.torproject.org/}{Tor} wird empfohlen (dabei eine andere Route als beim vorherigen Schritt benutzen).
\end{itemize}

\subsubsection{Account akkreditieren}
\begin{itemize}
 \item Der Client SOLL einen Akkreditierungsbrief mit dem öffentlichen Schlüssel $q$ drucken.
 \item Der Client SOLL den Akkreditierungsbrief NICHT veröffentlichen. 
 \item Der Client SOLL sich auf einer Mitgliederversammlung akkreditieren.
 \item Der Client MUSS den Akkreditierungsbrief in die Urne werfen. (siehe Abschnitt \ref{sec:urne_einwerfen})
\end{itemize}

\subsubsection{Nachricht verifzieren}
Dies macht der Client um die Nachrichten (z.B. Stimmabgaben anderer Nutzer auf ihre Authentizität zu überprüfen)
\begin{itemize}
 \item Der Client SOLL sich den Hash der offiziellen Liste der verifizierten Accounts besorgen (z.B. als Wahlbeobachter bei der Auszählung der Stimmzettel, oder über ein WebOfTrust)
 \item Der Client SOLL die offizielle Liste der verifizierten Accounts herunterladen und mit dem Hash auf Modifikationen überprüfen.
 \item Der Client SOLL prüfen, ob der Hash des Autors auch in der offiziellen Liste der verifizierten Accounts enthalten ist.
 \item Der Client SOLL prüfen, ob die Nachricht gültig gemäß der offiziellen Liste der registrierten Accounts vom Account-Server ist.
\end{itemize}

\subsection{Komponente: Urne}
\subsubsection{Akkreditierungsbrief in die Urne werfen} \label{sec:urne_einwerfen}
\begin{itemize}
 \item Die Akkreditierungskommission MUSS überprüfen, ob der Client stimmberechtigt ist.
 \item Die Akkreditierungskommission MUSS überprüfen, dass noch kein Vermerk existiert, dass der Client schon einen Akkreditierungsbrief in die Urne geworfen hat.
 \item Ist der Client stimmberechtigt und hat noch keinen Akkreditierungsbrief in die Urne geworfen, so MUSS die Akkreditierungskommission dem Client erlauben den Akkreditierungsbrief in die Urne zu werfen.
 \item Der Client MUSS den Akkreditierungsbrief in die Urne werfen.
 \item Die Akkreditierungskommission MUSS vermerken, dass der Client schon einen Akkreditierungsbrief in die Urne geworfen hat.
\end{itemize}


\end{document}

%\bibliographystyle{plain}
%\bibliography{../thermo}

\end{document}