\documentclass[a4paper,10pt]{article}

\usepackage[utf8]{inputenc}
\usepackage[ngerman]{babel}
\usepackage{amsmath}
\usepackage{amssymb}
\usepackage{fancyhdr}
%\usepackage{graphicx}
\usepackage{geometry}
\usepackage{enumerate}
%\usepackage{listings}
%\usepackage[rflt]{floatflt}
%\usepackage{tikz}
%\usepackage{wrapfig}
%\usepackage[standard, thmmarks]{ntheorem}
%\usepackage{algorithm}
%\usepackage{algorithmic}
%\usepackage{algorithm2e}
%\usepackage{subfigure}
%\usepackage{mathbbol}
\usepackage{listings}
\usepackage[pdfborder={0 0 0}]{hyperref}


%\lstset{numbers=left, numberstyle=\tiny, basicstyle=\footnotesize, numbersep=5pt}
%\lstset{extendedchars,inputencoding=utf8}
%\lstset{tabsize=2,breaklines}
%\lstset{language=}
%\renewcommand{\lstlistingname}{Quelltext}

\geometry{left=2cm, top=3cm, right=2cm, bottom=2.5cm}

%\renewcommand{\headrulewidth}{0.4pt} 
%\renewcommand{\footrulewidth}{0.4pt}

\pagestyle{fancy}

%\newcommand{\qed}{\hfill \ensuremath{\Box}}

\newcommand{\wiki}[2]{\href{http://wiki.piratenpartei.de/#1}{#2}}
\newcommand{\mytitle}{ Die Akkreditierungsurne }

\newcommand{\im}{\mathrm{im}\;}
\newcommand{\vol}{\text{vol}}
\newcommand{\conv}{\text{conv}}
\newcommand{\lin}{\text{lin}}
\newcommand{\cone}{\text{cone}}
\newcommand{\aff}{\text{aff}}
\newcommand{\rec}{\text{rec}}
\newcommand{\lineal}{\text{lineal}}
\newcommand{\eps}{\varepsilon}
\newcommand{\face}{\text{face}}
\newcommand{\row}{\text{row}}
\newcommand{\col}{\text{col}}
\newcommand{\sign}{\text{sign}}
\newcommand{\proj}{\mathrm{proj}}
\newcommand{\dom}{\mathrm{dom}\;}

\newcommand{\mat}[2]{\left( \begin{array}{#1}#2\end{array} \right)}

\lstset{language=XML}
\lstset{frame=single, breaklines}


%opening
\title{\mytitle}
\author{AG Liquid Democracy, Ansprechpartner: \wiki{Benutzer:Wobble}{Wobble}, \wiki{Benutzer:Dunkelzahn}{Dunkelzahn}, \wiki{Benutzer:moonopool}{moonopool}}

\lhead{Vorschlag der AG Liquid Democracy}
\chead{\mytitle}
\rhead{\today}

%\theoremstyle{nonumberplain}
% \newtheorem{claim}{Claim}[section]
% \newtheorem{observation}{Observation}[section]
% \newtheorem{problem}{Problem}[section]
% \newtheorem{conjecture}{Conjecture}[section]

\begin{document}

\parindent 0em
\parskip 0.5em

% Umfang: 5 Seiten!

\maketitle

\tableofcontents

\section{Einleitung}
Dieses Dokument definiert formal welche Komponente wie auf Anfragen etc. reagieren muss bzw. kann. In Abschnitt \ref{sec:datentypen} werden die Datentypen, die zum Informationsaustausch genutzt werden definiert. In Abschnitt \ref{sec:verfahren} werden die einzelnen Verfahrensschritte definiert. 

Dabei werden folgende Schlüsselworte (in Großbuchstaben) genutzt:
\begin{itemize}
 \item MUSS beschreibt eine Anforderung, die unbedingt zu erfüllen ist. Verletzungen von ``MUSS-Anforderungen`` müssen nicht verhinderbar aber durch dritte dokumentierbar sein.
 Eine Nichterfüllung kann zur Bestrafung des Betreibers der Komponente führen.
 \item SOLL beschreibt eine Anforderung, die für einen ordnungsgemäßen Ablauf erforderlich ist. Eine Verletzung einer ``SOLL-Anforderung'' muss nicht durch dritte dokumentierbar sein, wird aber in der Regel zu einer Fehlermeldung in einem Nachfolgeschritt des Ablaufes führen. Verletzungen von ``SOLL-Anforderungen'' sind nicht zu bestrafen.
 \item SOLL NICHT beschreibt eine Aktion, durch deren Umsetzung Eigenschaften des Verfahrens für den Komponentenbetreiber verloren gehen aber keine Auswirkungen auf andere Komponenten haben. Verletzungen von ``SOLL NICHT-Anforderungen'' sind nicht zu bestrafen.
 \item DARF NICHT beschreibt eine Aktion, die wesentliche Eigenschaften des Systems gefährdet. Verletzungen von ``DARF NICHT''-Anforderungen können bestraft werden.
\end{itemize}


\section{Datentypen} \label{sec:datentypen}
\subsection{Base32Encoding eines Byte-Array}
Gegeben ein Byte-Array $b$. Sei $s$ das Base32 encoding von $b$ gemäß \href{https://www.ietf.org/rfc/rfc4648.txt}{www.ietf.org/rfc/rfc4648.txt}. Sei $w$ der String, den man erhält, wenn man bei $s$ jedes '=' durch '\_' ersetzt. 
Dann ist $\_w$ definiert als das Base32Encoding von $b$.

TODO: Change Implementation

\paragraph{Begründung}
Die Hashes der öffentlichen Schlüssel werden als Identifikationsmerkmal des Autors genutzt. Dieses Identifikationsmerkmal wird in ID-Attributen benutzt und muss daher ein \href{http://www.w3.org/TR/xml11/#NT-Name}{bestimmtes Format} haben.

Mit diesem Encoding hat ein 512bit String eine Länge von 105 Zeichen, kann also getwittert werden.

\subsection{Liste verifizierter Accounts}
Jede Liste verifizierter Accounts validiert gegen folgendes XML-Schema:
\lstinputlisting{../xsd/legitimated.xsd}

Jeder Hash, dessen Base32Encoding als id-Attribut eines Account-Elements in einer Liste verifizierter Accounts $L$  eingetragen ist, gilt als \emph{verifiziert nach $L$}.

\subsection{Öffentlicher Schlüssel} \label{sec:publicKey}
Ein Öffentlicher Schlüssel im Sinne dieser Spezifikation ist ein \texttt{KeyInfo} Element gemäß \url{http://www.w3.org/TR/xmldsig-core/#sec-KeyInfo}, welches genau ein \texttt{KeyValue} Element enthält und sonst keine weiteren Elemente. Das \texttt{KeyValue} Element muss einen DSA-Schlüssel (gemäß \url{http://www.w3.org/2000/09/xmldsig#DSAKeyValue}) codieren.

\subsubsection{Hash eines öffentlichen Schlüssels}
Sei $s$ die exclusive canonicalization (gemäß \url{http://www.w3.org/TR/2002/REC-xml-exc-c14n-20020718/}) ohne InclusiveNamespaces des \texttt{KeyValue} Elements.

Der Hash des öffentlichen Schlüssels ist das Base32Encoding des SHA-512-Hashes (gemäß \url{http://csrc.nist.gov/publications/fips/fips180-2/fips180-2withchangenotice.pdf}) von $s$.

TODO: Change implementation

\subsection{Liste registrierter Accounts}
Jede Liste registrierter Accounts validiert gegen folgendes XML-Schema:
\lstinputlisting{../xsd/registered.xsd}

Jeder Hash, dessen Base32Encoding als id-Attribut eines gültigen Account-Elements in einer Liste registrierter Accounts $L$  eingetragen ist, gilt als \emph{registriert nach $L$}. 
Ein Account-Element ist gültig, falls
\begin{itemize}
 \item das enthaltene \texttt{sig:KeyInfo} Element ein öffentlicher Schlüssel gemäß Abschnitt \ref{sec:publicKey} ist.
 \item der Hash des öffentlichen Schlüssels mit dem Wert des id-Attributs übereinstimmt.
\end{itemize}

\subsection{Nachricht}
Jede Nachricht validiert gegen folgendes XML-Schmea:
\lstinputlisting{../xsd/message.xsd}

Eine Nachricht ist \emph{gültig} gegenüber einer Liste registrierter Accounts $R$, falls
\begin{itemize}
 \item der Hash \texttt{/Message/Body@author} nach $R$ registriert ist.
 \item Die Signatur (\texttt{/Message/Signature}) gegen den öffentlichen Schlüssel, der in $R$ für den Hash \texttt{/Message/Body@author} gespeichert ist, verifiziert.
 \item Die Signatur (\texttt{/Message/Signature}) genau eine Referenz enthält, wobei
 \begin{itemize}
  \item die URI der Referenz genau \texttt{"/Message/Body"} ist.
  \item die Referenz keine expliziten Transformationen benennt
  \item \href{http://www.w3.org/2001/04/xmlenc#sha512}{SHA-512} als DigestMethod benutzt.
 \end{itemize}
\end{itemize}

Eine Nachricht ist \emph{verifiziert} gegenüber einer Liste verifizierter Accounts $V$, falls der Hash \texttt{/Message/Body@author} nach $V$ verifiziert ist.

TODO:change implementation

\subsection{Account-Erstellungs-Nachricht}
Eine Nachricht ist eine Account-Erstellungs-Nachricht, falls \texttt{/Message/Body} genau ein Element enthält, welches ein öffentlicher Schlüssel (nach Abschnitt \label{sec:publickKey}) ist. 
Die Account-Erstellungs-Nachricht ist gültig, 
\begin{itemize}
 \item falls \texttt{/Message/Body@author} mit dem Hash des öffentlichen Schlüssels übereinstimmt.
 \item Die Signatur (\texttt{/Message/Signature}) gegen den im Body gespeicherten öffentlichen Schlüssel verifiziert.
 \item Die Signatur (\texttt{/Message/Signature}) genau eine Referenz enthält, wobei
 \begin{itemize}
  \item die URI der Referenz genau \texttt{"/Message/Body"} ist.
  \item die Referenz keine expliziten Transformationen benennt
  \item \href{http://www.w3.org/2001/04/xmlenc#sha512}{SHA-512} als DigestMethod benutzt.
 \end{itemize}
\end{itemize}

\subsubsection{Hash der Nachricht}
Der Hash der Nachricht ist der Wert des Attributs \texttt{/Message/Body@author}.

\subsubsection{Schlüssel der Nachricht}
Der Schlüssel der Nachricht ist das Element \texttt{/Message/Body/KeyInfo}.

\subsection{Akkreditierungsbrief}
Der Akkreditierungsbrief ist ein auf Papier gedruckter Hash des öffentlichen Schlüssels. Konkret ist enthalten:
\begin{itemize}
 \item Der Hash als Text in Base32Encoding
 \item Der Hash als QR-CODE (TODO:genauer)
\end{itemize}


\section{Verfahren} \label{sec:verfahren}
Das System ist in die folgenden Komponenten aufgeteilt:
\begin{itemize}
 \item Verification-Server: Dieser stellt eine Liste der verifizierten Hashes bereit
 \item Account-Server: Dieser verwaltet die Liste aller registrierten Accounts
 \item Urne: Die Urne dient der Sammlung der Hashes der öffentlichen Schlüssel
 \item Client: Jeder Pirat betreibt einen Client
\end{itemize}

\subsection{Komponente: Verification-Server}
Der Verification-Server verwaltet die offizielle Liste der verifizierten Accounts, welche durch den Urnengang verifiziert wurde.
\subsubsection{statisches Verhalten}
\begin{itemize}
 \item Der Verification-Server MUSS auf einem wohldefinierten Ort die offizielle Liste der verifizierten Accounts veröffentlichen.
\end{itemize}


\subsection{Komponente: Account-Server}
Der Account-Server verwaltet die offizielle Liste der registrierten Accounts, welche sich bis jetzt registriert haben.
\subsubsection{statisches Verhalten}
\begin{itemize}
 \item Der Account-Server MUSS auf einem wohldefinierten Ort die offizielle Liste der registrierten Accounts veröffentlichen.
\end{itemize}

\subsubsection{Empfang einer Nachricht $m$ }
\begin{itemize}
 \item Der Account-Server MUSS prüfen, ob $m$ eine gültig Account-Erstellungs-Nachricht ist.
 \item Ist $m$ eine gültige Account-Erstellungs-Nachricht, so MUSS der Account-Server in der offizielle Liste der registrierten Accounts einen Eintrag unter dem Hash von $m$ mit dem Schlüssel von $m$ erstellen. Falls schon ein Eintrag mit dem Hash von $m$ existiert, so MUSS der existierende Eintrag überschrieben werden.
 \item Ist $m$ keine gültige Account-Erstellungs-Nachricht, so DARF die offizielle Liste der registrierten Accounts NICHT modifiziert werden.
\end{itemize}

\paragraph{Begründung}
Nachrichten müssen auf Gültigkeit geprüft werden, damit dritte nicht die Account-Details Fremder überschreiben.

Durch den öffentlich sichtbaren Effekt der Nachricht (Aktualisierung der registrierten Accounts), ist keine weitere besondere Antwort-Nachricht nötig. Mmh... dann sollte diese offizielle Liste vom Account-Server signiert werden.


\subsection{Komponente: Client}
Mit dem Client können neue Accounts erstellt werden und Nachrichten versendet werden.
\subsubsection{Account erstellen}
\begin{itemize}
 \item Der Client erstellt einen neues Schlüsselpaar mit privaten Schlüssel $p$ und öffentlichen Schlüssel $q$.
 \item Der Client SOLL das Schlüsselpaar auf einem privaten aber zugänglichen Ort speichern.
 \item Der Client SOLL den privaten Schlüssel NICHT veröffentlichen.
 \item Der Client SOLL den öffentlichen Schlüssel NICHT in einer Form veröffentlichen, die es ermöglicht Schlussfolgerungen auf den Inhaber des öffentlichen Schlüssels zu ziehen.
\end{itemize}

\subsubsection{Account registrieren}
\begin{itemize}
 \item Der Client SOLL eine Account-Erstellungs-Nachricht aus dem öffentlichen Schlüssel $q$ erstellen und dem Account-Server senden. Die Nutzung von \href{https://www.torproject.org/}{Tor} wird empfohlen.
 \item Der Client SOLL überprüfen, ob der Account-Server die öffentliche Datenbank aktualisiert hat und eine Kopie speichern. Die Nutzung von \href{https://www.torproject.org/}{Tor} wird empfohlen (dabei eine andere Route als beim vorherigen Schritt benutzen).
\end{itemize}

\subsubsection{Account akkreditieren}
\begin{itemize}
 \item Der Client SOLL einen Akkreditierungsbrief mit dem öffentlichen Schlüssel $q$ drucken.
 \item Der Client SOLL den Akkreditierungsbrief NICHT veröffentlichen. 
 \item Der Client SOLL sich auf einer Mitgliederversammlung akkreditieren.
 \item Der Client MUSS den Akkreditierungsbrief in die Urne werfen. (siehe Abschnitt \ref{sec:urne_einwerfen})
\end{itemize}

\subsubsection{Nachricht verifzieren}
Dies macht der Client um die Nachrichten (z.B. Stimmabgaben anderer Nutzer auf ihre Authentizität zu überprüfen)
\begin{itemize}
 \item Der Client SOLL sich den Hash der offiziellen Liste der verifizierten Accounts besorgen (z.B. als Wahlbeobachter bei der Auszählung der Stimmzettel, oder über ein WebOfTrust)
 \item Der Client SOLL die offizielle Liste der verifizierten Accounts herunterladen und mit dem Hash auf Modifikationen überprüfen.
 \item Der Client SOLL prüfen, ob der Hash des Autors auch in der Liste der verifizierten Accounts enthalten ist.
 \item Der Client SOLL prüfen, ob die Nachricht gültig gemäß der offiziellen Liste der registrierten Accounts vom Account-Server ist.
\end{itemize}

\subsection{Komponente: Urne}
\subsubsection{Akkreditierungsbrief in die Urne werfen} \label{sec:urne_einwerfen}
\begin{itemize}
 \item Die Akkreditierungskommission MUSS überprüfen, ob der Client stimmberechtigt ist.
 \item Die Akkreditierungskommission MUSS überprüfen, dass noch kein Vermerk existiert, dass der Client schon einen Akkreditierungsbrief in die Urne geworfen hat.
 \item Ist der Client stimmberechtigt und hat noch keinen Akkreditierungsbrief in die Urne geworfen, so MUSS die Akkreditierungskommission dem Client erlauben den Akkreditierungsbrief in die Urne zu werfen.
 \item Der Client MUSS den Akkreditierungsbrief in die Urne werfen.
 \item Die Akkreditierungskommission MUSS vermerken, dass der Client schon einen Akkreditierungsbrief in die Urne geworfen hat.
\end{itemize}


\end{document}

%\bibliographystyle{plain}
%\bibliography{../thermo}

\end{document}